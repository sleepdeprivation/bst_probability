\documentclass[10pt,a4paper]{article}
\usepackage[utf8]{inputenc}
\usepackage{amsmath}
\usepackage{amsfonts}
\usepackage{amssymb}
\author{Christian Burke}
\begin{document}
With $0 < k < n$ and $k,n \in \mathbb{Z}$, we prove the following lemmas
\section*{Lemma 1}
$$ \prod_{i=k}^n i = \frac{n!}{(k-1)!}$$
Proof:
$$ \prod_{i=k}^n i = \prod_{i=k}^{n} i \frac{(k-1)!}{(k-1)!}
 = \prod_{i=k}^n i \frac{\prod_{i=1}^{k-1} i}{(k-1)!}
 = \frac{n!}{(k-1)!}$$
\section*{Lemma 2}
$$ \prod_{i=0}^k (n-i) = \frac{n!}{(n-k-1)!}$$
Proof:
$$ \prod_{i=0}^k n-i = (n)(n-1)\cdots(n-k) = \prod_{i=(n-k)}^n i$$
And by using lemma 1 we get
$$ \prod_{i=(n-k)}^n i = \frac{n!}{(n-k-1)!}$$
\section*{Problem Statement}
We are given a set of $n$ nodes in a valid binary search tree $B$, with unique keys from a (finite) set of contiguous integers $S$.\\
We then pick another key $x \in S$, and begin searching for the key using a linear BST algorithm.\footnote{Is 'a linear BST' algorithm guaranteed to traverse nodes in a certain order? Can we prove that the linear BST algorithm is unique given certain constraints? (eg "nodes are traversed in same order")}
How does $p(x \in B)$ change with each node traversed?\\
\section*{Calculating $p(x \in B)_0$ using products}
Before beginning the search, ie at time $t=0$, we calculate $p(x \in B)_0$ as the probability that $x$ is chosen in $n$ tries from a set of $|S|$ items, by first calculating its inverse.\\
The probability of not picking $x$ on the $1$st try is $\frac{|S|-1}{|S|}$. On the next try, it will be $\frac{|S|-2}{|S|-1}$, accounting for the missing element, and so on until we have the final probability as $\frac{|S|-n}{|S| - (n-1)}$.\\
Thus the probability that we never picked $x$ for any node is:
$$p(x \not \in B)_0 = \prod_{i = 0}^{n-1} \frac{|S|-(i+1)}{|S| - i} = \frac{\prod_{i = 0}^{n-1} (|S|-1)-i}{\prod_{i = 0}^{n-1} |S| - i}$$
We can now apply lemma 2 on both numerator and denominator to get
$$
\bigg(\frac{(|S|-1)!}{((|S|-1) - (n-1) - 1)!}\bigg)
\bigg(\frac{|S|!}{(|S| - (n-1) - 1)!}\bigg)^{-1}
$$
$$
= 
\frac{|S|-1!}{(|S|-n-1)!}
\frac{(|S|-n)!}{|S|!}
= \frac{|S|-n}{|S|}
$$
Which finally gives us\\
$$p(x \in B)_0 = (1-p(x \not \in B)_0) = \frac{n}{|S|}$$

\section*{Calculating $p(x \in B)_0$ using binomial coefficients}

There are $|S|-1 \choose n-1$ ways to pick $x$, and then $n-1$ more elements. There are, in total, $|S| \choose n$ ways to pick $n$ elements. Thus,

$$p(x \in B)_0
	= 
	\frac{
			{|S| - 1 \choose n-1}
		}{
			{|S| \choose n}
		}
= 	
		\frac{(|S|-1)!}{(n-1)!((|S|-1) - (n-1))!}
	\div
		\frac{|S|!}{n!(|S|-n)!}
$$
$$=	
		\frac{(|S|-1)!}{(n-1)!(|S|-1 - n + 1)!}
	\times
		\frac{n!(|S|-n)!}{|S|!}
=
		\frac{(|S|-1)!}{(|S| - n)!}
	\times
		\frac{(|S|-n)!}{|S|!}
$$
$$=
	\frac{
		(|S|-1)!n!
	}{
		|S|! (n-1)!
	} = \frac{n}{|S|}
$$
\section*{Some notation}
For a given node $B_i$ in the tree $B$,\\
$b_i$ is the value at that node,\\
$|B_i|$ is the size of the sub-tree rooted at $B_i$.
\section*{Calculating $p(x \in B)_1$}
After checking the root node, which we will call $B_1$, we will have either found $x$, or we will not have. If we found $x$, we now know $p(x \in B)_1 = 1$, so we will consider the other case.\\
We must now calculate $p(x \in B)_1 = p(x \in B | b_1 \neq x)$.\\
Because the set $B$ is a binary tree, the search algorithm will disregard one of $B_1$'s two children, reducing the size of the domain. We'll call $B_1$'s children $B_L$ and $B_R$. Since all the values here are arbitrary, we can say without loss of generality that the algorithm will pick $B_L$ as the next node to examine, which we will now label $B_2$, and further, we can claim that the algorithm has found $x > b_1$, since it would be the same either way.
\footnote{If such a bold claim makes you uncomfortable, we can write $x \oplus b_2$ where $\oplus \in \{<, >\}$}\\
We have now ruled out every element in the set $\{b_i | b_i <= b_1\}$, and we are ready for our calculation. Let's call the new domain $S_1 = S - \{b_i | b_i <= b_1 \}$. We can use the previous result to obtain
$$p(x \in B)_1 = \frac{|B_2|}{|S_1|}$$
That is, the size of the sub-tree $B_L$, divided by the size of the reduced domain $S_1$.
\section*{In general, $p(x \in B)_i$}
We now have a general pattern. The algorithm will examin a sequence of $k$ nodes $\{B_1, B_2, B_3 \cdots B_k\}$, as it does this, it will generate subsets of $S$ according to the following:
$$S_n = S_{n-1} - \{b_i | b_i <= b_1 \}$$.
\section*{GARBAGE}
The probability of not picking $X$ is thus
$$p(x \not \in B)_0 = \prod_{i=0}^{n-1} \bigg[ \frac{|S|-(i + 1)}{|S| - i} \bigg]
= \frac{\prod_{i=0}^{n-1} |S|- (i+1)}{\prod_{i=0}^{n-1} |S| - i}
= \frac{\prod_{i=0}^{n-1} (|S|-1)-i}{\prod_{i=0}^{n-1} |S| - i}$$
$$= \frac{(|S|-1)!}{((|S| - 1) - n - 1)!} \frac{(|S| - n - 1)!}{|S|!}$$
$$= \frac{(|S|-1)!(|S| - n - 1)!}{(|S| - n - 2)!|S|!}$$
---\\
We can calculate the probability of picking $x$ by,
$$p(x \in B)_0
=\prod_{i=0}^{n-1}\frac{1}{ |S| - i}
=\frac{1}{\prod_{i=0}^{n-1} |S| - i}
= \frac{(|S|-(n-1)-1)!}{|S|!}$$
$$ = \frac{(|S|- n)!}{|S|!}
=\frac{1}{|S|!/(n-1)!} = \frac{(n-1)!}{|S|!}$$
and through similar reasoning we can determine
\end{document}