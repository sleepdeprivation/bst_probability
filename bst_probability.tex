\documentclass[10pt,a4paper]{article}
\usepackage[utf8]{inputenc}
\usepackage{amsmath}
\usepackage{amsfonts}
\usepackage{amssymb}
\author{Christian Burke}
\begin{document}
With $0 < k < n$ and $k,n \in \mathbb{Z}$, we prove the following lemmas
\section*{Lemma 1}
$$ \prod_{i=k}^n i = \frac{n!}{(k-1)!}$$
Proof:
$$ \prod_{i=k}^n i = \prod_{i=k}^{n} i \frac{(k-1)!}{(k-1)!}
 = \prod_{i=k}^n i \frac{\prod_{i=1}^{k-1} i}{(k-1)!}
 = \frac{n!}{(k-1)!}$$
\section*{Lemma 2}
$$ \prod_{i=0}^k (n-i) = \frac{n!}{(n-k-1)!}$$
Proof:
$$ \prod_{i=0}^k n-i = (n)(n-1)\cdots(n-k) = \prod_{i=(n-k)}^n i$$
And by using lemma 1 we get
$$ \prod_{i=(n-k)}^n i = \frac{n!}{(n-k-1)!}$$
\section*{Problem Statement}
We are given a set of $n$ nodes in a valid binary search tree $B$, with unique keys from a (finite) set of contiguous integers $S$.\\
We then pick another key $x \in S$, and begin searching for the key using a linear BST algorithm.\footnote{Is 'a linear BST' algorithm guaranteed to traverse nodes in a certain order? Can we prove that the linear BST algorithm is unique given certain constraints? (eg "nodes are traversed in same order")}
How does $p(x \in B)$ change with each node traversed?\\
\section*{Calculating $p(x \in B)_0$}
Before beginning the search, ie at time $t=0$, we calculate $p(x \in B)_0$ as the probability that $x$ is chosen in $n$ tries from a set of $|S|$ items.\\
The probability of picking an element that is not $x$ on the $1$st try is $\frac{1}{|S|}$. On the next try, it will be $\frac{1}{|S|-2}$, accounting for the missing element, and so on until we have the final probability as $\frac{1}{|S|- n + 1}$.\\
We can calculate the probability of picking $x$ by,
$$p(x \in B)_0
=\prod_{i=0}^{n-1}\frac{1}{ |S| - i}
=\frac{1}{\prod_{i=0}^{n-1} |S| - i}
= \frac{(|S|-(n-1)-1)!}{|S|!}$$
$$ = \frac{(|S|- n)!}{|S|!} $$
Finally we can confirm:
$$1 - \frac{|S|!}{(|S|-n-1)!} = \frac{(|S|-n-1)! + |S|!}{(|S|-n-1)!}$$

\section*{A different method for calculating the initial probability}

There are $|S|-1 \choose n-1$ ways to pick $x$, and then $n-1$ more elements. There are, in total, $|S| \choose n$ ways to pick $n$ elements. Thus,

$$p(x \in B)_0
	= 
	\frac{
			{|S| - 1 \choose n-1}
		}{
			{|S| \choose n}
		}
= 	
		\frac{(|S|-1)!}{(n-1)!(|S|-1 - (n-1))!}
	\div
		\frac{|S|!}{n!(|S|-n)!}
$$
$$=	
		\frac{(|S|-1)!}{n!(|S|-1 - n + 1)!}
	\times
		\frac{n!(|S|-n)!}{|S|!}
=
		\frac{(|S|-1)!}{(|S| - n)!}
	\times
		\frac{(|S|-n)!}{|S|!}
$$
$$=
	\frac{
		(|S|-1)! (|S|-n)!
	}{
		(|S|-1 - n)! |S|!
	}
$$

\section*{GARBAGE}
The probability of not picking $X$ is thus
$$p(x \not \in B)_0 = \prod_{i=0}^{n-1} \bigg[ \frac{|S|-(i + 1)}{|S| - i} \bigg]
= \frac{\prod_{i=0}^{n-1} |S|- (i+1)}{\prod_{i=0}^{n-1} |S| - i}
= \frac{\prod_{i=0}^{n-1} (|S|-1)-i}{\prod_{i=0}^{n-1} |S| - i}$$
$$= \frac{(|S|-1)!}{((|S| - 1) - n - 1)!} \frac{(|S| - n - 1)!}{|S|!}$$
$$= \frac{(|S|-1)!(|S| - n - 1)!}{(|S| - n - 2)!|S|!}$$
and through similar reasoning we can determine
\end{document}